\documentclass[italian,a4paper,12pt,oneside]{report}
\usepackage[T1]{fontenc}
\usepackage[utf8]{inputenc}
\usepackage{geometry}
\usepackage{graphicx}
\usepackage{forest}
\setcounter{secnumdepth}{3}
\setcounter{tocdepth}{1}
\usepackage{float}
\usepackage{babel}
\usepackage{microtype} % migliora espansione dei font. Suggerito su ``L'arte di scrivere in LaTeX, pag. 45''
\usepackage{indentfirst} % indentazione anche su primo paragrafo.  Suggerito su ``L'arte di scrivere in LaTeX, pag. 45''

\title{Appunti per lezione sicurezza informatica\\
Revisione 0.01}
\author{Germano Massullo - germano.massullo@gmail.com}
\date{}

\begin{document}

\maketitle

Linee guida (ad uso personale) per discorso della lezione ``pillole di sicurezza
informatica'', rivolto ad un pubblico generalista, dove formalismi e tecnicismi
sono stati eliminati per favorire al massimo l'apprendimento degli argomenti
illustrati.
\textbf{Il presente documento non è da considerarsi una trascrizione del ``parlato''
della presentazione}.



\chapter{Sicurezza dei dispositivi elettronici}
I dispositivi elettronici ormai pervadono la nostra quotidianetà e la influenzano
in modo diretto. Internet non è più quel luogo isolato che non influenzava in alcun
modo il mondo reale, ora ciò che accade sulla rete influisce anche sulle nostre vite.
Le tecnologie hanno accorciato le distanze tra le persone, semplificato
le nostre attività di routine. Il rovescio della medaglia è che hanno assottigliato
la sfera della nostra privacy rendendoci più vulnerabili ad abusi. È perciò
importante fare un uso consapevole di questi oggetti, ovvero essere informati
sui potenziali rischi che si corrono ed occorre assicurarsi di avere il controllo esclusivo
di questi dispositivi.

\section{integrità dei dati}
Spesso le persone pongono una illimitata fiducia circa l'affidabilità dei
propri dispositivi elettronici.
Il tempo medio prima di un guasto, in inglese Mean Time Before Failure, è il
lasso di tempo previsto dal produttore prima dell'occorrenza di un malfunzionamento del
dispositivo. È una delle specifiche di progetto di ogni prodotto commercializzato.
Se si è fortunati il prodotto non si romperà prima del MTBF previsto, in caso
contrario si ricadrà nella seguente distribuzione di probabilità di rotture che
ha la forma di una vasca da bagno. Qui, si può notare come i dispositivi
siano soggetti ad un alto tasso di rotture durante i primi periodi della loro
vita utile, poi il tasso di rotture diminuisce vertiginosamente, ed infine
aumenta nuovamente verso il fine ciclo di vita del prodotto.
\subsection{Backup}
È pertanto importante predisporre dei backup periodici dei propri dati.
Almeno due backup, e se possibile alloggiarli in posizioni diverse, così
da mettersi al sicuro anche da eventuali furti in casa.
Non utilizzare mai penne USB in quanto sono i prodotti elettronici più
inaffidabili a causa dell'impiego di materiali di infima qualità.

\section{distruzione dischi}
Quando cancellate un file, il computer marcherà tale file come cancellato, non
lo cancellerà veramente finché non vi scriverà dei nuovi dati sopra quelli vecchi
Se volete veramente cancellare i dati in maniera definitiva, utilizzate i
seguenti software.

\section{Il mio computer è lento}
Spesso le persone si lamentano, affermando che il loro computer è molto lento. Ai
giorni d'oggi, i computer hanno una potenza che eccede le esigenze di
uso quotidiano, pertanto spesso e volentieri ogni pesante rallentamento
durante il normale utilizzo, è indice che il computer è infetto da un qualche
tipo di malware. Per comodità oggi chiameremo con il termine ``virus'' ogni tipo
di software installato da malintenzionati. Installare un antivirus su una
macchina infetta non risolverà il problema: è come con i vaccini, non si può
somministrare un vaccino ad una persona che presenta già una patologia, pertanto
in parole povere: chi arriva prima vince.

\subsection{Virus}
Computer infetti, spesso costituiscono le cosiddette botnet. Esse servono a
gruppi di malintenzionati per i più disparati utilizzi: occultare le proprie
tracce per effettuare truffe online, mandare e-mail di spam, infettare altri
computer, ecc. Dato che in caso di attacchi informatici provenienti da propri
dispositivi, ci si ritroverà a doversi difendere in tribunale, è bene prevenire
piuttosto che curare.

\section{Buffer overflow}
Questo è un processore, la cosiddetta CPU. È il dispositivo che all'interno
di computer e cellulari svolge tutte le operazioni necessarie al funzionamento
della macchina. Normalmente il processore lavora nella seguente maniera, ma a
noi in questa occasione non interessa scendere a questo livello di dettaglio.

Perchè è importante parlare del processore? 
I processori operano gestendo celle
di memoria finita. Quando un programmatore crea un programma, deve assegnare ad
ogni routine una determinata area di memoria, a suo piacimento.
Quando poi il programma userà quell'area di memoria, il programmatore dovrà
assicurarsi che il programma non usi accidentalmente più memoria di quella
assegnata. Un esempio pratico: quando effettuate una registrazione ad un
sito web, generalmente vi viene detto che di inserire una password non più
lunga di 20 caratteri. Quindi si presume che il programmatore abbia allocato
un'area di memoria di 20 caratteri per gestire la vostra password. Che cosa
potrebbe accadere inserendo una password più lunga di 20 caratteri?
Se il programmatore ha fatto il suo dovere, sarà per voi impossibile inserire
più di 20 caratteri. Se invece è stato pigro, la password che state inserendo
forzerà l'area di memoria di 20 caratteri, dilagando nel programma. Immaginatelo
come un cassone d'acqua riempito oltre l'orlo e che inizia a trasbordare liquido.

Fino a qualche anno fa, generalmente i virus venivano attivati manualmente
mediante l'attivazione accidentale da parte della vittima, di file eseguibili
formato exe. Ora le tecniche di attacco si sono fatte più raffinate ed i virus
si propagano sfruttando dei bachi presenti nei sistemi delle vittime, come
quello accennato precedentemente.


Nel modello di funzionamento da noi ipotizzato, non corrispondente alla realtà,
i computer durante il loro funzionamento processano linee di codice così
strutturate: istruzione macchina concatenata al contatore di riga.
Il contatore di riga istruisce la macchina su quale riga di comando eseguire
alla successiva passata. Assumiamo che il program counter sia un'area di memoria
adiacente a quella che gestisce le righe dei comandi del programma. Inoltre
esso viene aggiornato automaticamente dalla CPU.

Immaginiamo di usare il seguente programma che gestisce le mele disponibili:
all'inizio l'utente inserisce il numero di mele, che qui assumiamo sia zero.

SEGUE DESCRIZIONE DEL CODICE DELLE SLIDE

Assumiamo che un malintenzionato riesca ad infilare nel pezzo finale del
programma, un pezzo di codice che distrugga il sistema. A noi non importa
come ce l'abbia messo, in quanto il problema non è inserire un'istruzione malevola
nella memoria, ma deviare il flusso di esecuzione del programma in maniera tale che
il processore esegua tale istruzione.

Dato che le dimensioni delle celle di memoria dedicate alla ricezione di istruzioni
hanno una dimensione fissata, qualora un malintenzionato riesca a forzare tali celle
facendo fuoriuscire del codice al di fuori di esse, probabilmente riuscirà
ad intaccare l'integrità del contatore di riga, facendo eseguire al sistema
delle righe di codice diverse da quelle che avrebbe dovuto eseguire.





\section{Smart tv che hanno webcam in camera da letto}
Le smart tv e tutti gli oggetti ``internet of things`` non subiranno alcun
aggiornamento una volta che la casa produttrice ha deciso che i clienti
debbano acquistarne una nuova versione. Ciò esporrà gli utilizzatori ad
una serie di problemi di sicurezza. Dispositivi di questo tipo integrano
webcam e microfoni, che sono una grave minaccia alla privacy soprattutto in luoghi
intimi come le camere da letto.

\section{Wi-Fi e smart home}
Attualmente assistiamo ad una grande diffusione dei dispositivi per la domotica,
le cosiddette smart home. Spesso e volentieri tali dispositivi sono collegati
ad internet con una connessione Wi-Fi. Anche se la nostra connessione Wi-Fi
è crittografata, non bisogna sentirsi completamente al sicuro.
Una persona nel raggio di azione della vostra abitazione può vedere quanti
e quali dispositivi sono connessi, vedere se effettuano traffico, e pertanto
fare una stima delle persone presenti in casa, ed agire di conseguenza nel
caso voglia commettere un furto. Inoltre, essendo possibile effettuare un
''attacco di disconnessione'', immaginate l'effetto che esso potrebbe avere
su un sistema di videosorveglianza. Per ovviare a questa minaccia basta munirsi
di un impianto collegato con il cavo di rete.



\chapter{Sicurezza nell'uso di internet}

\section{crittografia asimmetrica}
Quando vogliamo scambiare un file crittografato con un'altra persona, poi avremo
il problema di come inviargli la chiave di decodifica.
La crittografia asimmetrica sopperisce a questa necessità, in quanto ha origine
durante la guerra fredda, dal bisogno di potersi scambiare chiavi di
crittografia attraverso un canale di comunicazione non sicuro.
Prima dell'avvento della crittografia asimmetrica, ad esempio nella II
guerra mondiale, i sommergibili tedeschi partivano dalle basi con un libro
di tutte le chiavi di crittografia da usare giorno per giorno con il
dispositivo Enigma.

Nella crittografia asimmetrica ognuno dei due soggetti
coinvolti, possiede una chiave privata ed una chiave pubblica.
La chiave pubblica viene generata dalla chiave privata attraverso operazioni matematiche.
Le chiavi pubbliche vengono pubblicate su elenchi pubblici, in quanto non contengono informazioni
sensibili. Affinché si possa essere sicuri che una chiave pubblica appartenga effettivamente ad una
persona, si possono adottare varie strategie:
\begin{itemize}
 \item farsela firmare da una serie di testimoni che confermino la tua identità;
 \item farsela firmare da un'autorità che certifica la tua identità
\end{itemize}
Una chiave pubblica non firmata ha lo stesso valore di una carta di identità
fatta da noi stessi su un tovagliolo
Quando Alice vuole mandare un messaggio crittografato a Bob, non userà la propria
chiave di crittografia, bensì prende la chiave pubblica di Bob e la utilizza per
crittografare il messaggio.
Quando Bob riceverà il dato crittografato, egli per decodificare non userà la sua chiave pubblica,
bensì la sua chiave privata. Per spiegarvi il motivi per il quale egli userà la sua chiave privata e
non quella pubblica, dovremmo fare varie lezioni di matematica che qui non abbiamo tempo di fare.
Inoltre Alice, per far sì che Bob sia sicuro che il messaggio sia stato
effettivamente spedito da lei, firmerà il tutto con la sua chiave privata.
Infine Bob andrà a vedere chi ha firmato la corrispondente chiave pubblica
della chiave privata di Alice

\section{Gestione password}
\subsection{KeepassXC}
KeepassXC è un programma che permette la generazione di password sicure
ed il loro stoccaggio nella più completa sicurezza. Pensate ad esso come una cassaforte
dove la combinazione è una unica password da voi scelta. Non dovrete più ricordare password
di ogni account in vostro possesso, vi basterà fare copia ed incolla da KeepassXC.
\subsection{Generazione password}
Devono avere una lunghezza di almeno 12 caratteri e contenere solo caratteri
casuali. Immaginate il gestore di password come una cassaforte, si dovrà solamente
ricordare la combinazione, che nel nostro caso sarà una password.
Per generare una password sicura il consiglio è prendere una frase di un libro
che ci piace, e per ogni parola prenderne la lettera iniziale.

Fare esempio del lucchetto per i numeri di tentativi possibili

\section{Furti di identità}
Le persone che non usano password robuste permettono a malintenzionati di
porre in essere la seguente truffa: un giorno uno dei vostri amici, vi contatta
da Facebook affermando di essere in viaggio all'estero e di esser rimasto senza
denaro a causa di uno scippo. Vi chiederà di mandargli una ingente somma di
soldi che gli permetteranno di poter tornare in Italia. In realtà il vostro
amico non si è mai mosso, e la persona che vi sta scrivendo è un impostore
che mira solamente a farsi dare i vostro soldi.
Per non parlar poi dei rischi per chi subisce un furto di identità: potrebbe
ritrovarsi intestati contratti per servizi mai richiesti, ammanchi nei conti
bancari, ecc.

\section{virus Zeus}
Il virus Zeus è un virus che permette ad un malintenzionato di accedere al
conto corrente della vittima anche in presenza di token hardware. Nel momento
stesso che la persona sta inserendo il numero generato dal tuo token, il virus
predispone un bonifico bancario che servirà a rubare il denaro.
Non usate mai il cellulare per predisporre operazioni di internet banking.

\section{Geolocalizzazione globale Google}
I telefoni Android permettono la localizzazione della persona anche all'interno
di edifici chiusi, e con estrema precisione. Ciò è possibile grazie
all'interpolazione di dati provenienti da:
\begin{itemize}
 \item segnale GPS;
 \item segnale antenna cellulari;
 \item segnale Wi-Fi.
\end{itemize}
Dato che gli algoritmi di Google non sono pubblici, ho elaborato una mia teoria
su come abbiano sviluppato una tecnologia del genere.
Quando si è in esterno, il telefono associa le variazioni di potenza del
segnale delle celle telefoniche ad una specifica area geografica grazie
all'ausilio dell'antenna GPS. Quando il telefono si troverà privo della
coperatura GPS (ad esempio in luoghi chiusi), sopperirà a tale mancanza
confrontando la potenza delle celle telefoniche con il database memorizzato
precedentemente. La stessa operazione si ripete utilizzando le reti Wi-Fi
incontrate durante passaggio in aree coperte da segnale GPS.

\section{protezione minori su internet}
Dopo varie ricerche online, sono giunto alla conclusione che Kaspersky for kids
sia la migliore scelta per chi volesse proteggere i propri figli dalle insidie
della rete. Tale software, installato su dispositivi mobili, permette anche di
monitorare i loro spostamenti in luoghi aperti. La presenza del software è
volutamente non occultabile. Una volta installato si deve avere l'accortezza
di configurarlo in maniera tale da renderlo non disinstallabile.

\section{nei servizi online gratuiti, sei tu il prodotto}
Google ha fondato GMail che forniva uno spazio di archiviazione di ~1 GB
quando tutte le caselle di posta elettronica ne fornivano uno di pochi MB.
Ora con Google Drive siamo giunti a spazi di 20 GB. Fornire una dimensione
di archiviazione così grande per miliardi di utenti, con un così alto
livello di ridondanza e tolleranza ai guasti, costa un mucchio di soldi.
Dato che le società non sono onlus, loro ricavano i propri profitti rivendendo
dati non aggregati circa usi ed abitudini dei propri clienti.

\end{document}
