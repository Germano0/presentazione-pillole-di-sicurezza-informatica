\documentclass[italian,aspectratio=169]{beamer}
\usetheme{Warsaw}
\usepackage[utf8]{inputenc}
\usepackage{adjustbox} % serve per regolare automaticamente l'ampiezza delle tabelle
\usepackage{algorithm,algpseudocode}
\usepackage{default}
\usepackage{verbatim}
\usepackage{tikz}

\author{Germano Massullo - germano.massullo@gmail.com}
\date{}
\title{Pillole di sicurezza informatica}

\begin{document}
\begin{frame}
 \maketitle
  \vspace{-2.5cm}
   \begin{center}
   \includegraphics{immagini/intro.jpg}
   \end{center}
\end{frame}

\begin{frame}
 \tableofcontents{}
\end{frame}

\section{Sicurezza dei dispositivi elettronici}
\begin{frame}
 \begin{center}
  \huge Sicurezza dei dispositivi elettronici
 \end{center}
\end{frame}

\begin{frame}
 \frametitle{Integrità dei dati}
 \begin{block}{Tempo medio tra i guasti}
  Il tempo medio tra i guasti (in inglese mean time between failures,
  spesso abbreviato in MTBF), è un parametro di affidabilità applicabile a
  dispositivi meccanici, elettrici ed elettronici e ad applicazioni software.
 \end{block}
 \begin{center}
   \includegraphics[scale=0.5]{immagini/bathtube_curve.jpg}
 \end{center}
\end{frame}


\begin{frame}
 \frametitle{Integrità dei dati}
 \begin{center}
   \huge almeno due backup!
 \end{center}
 \begin{center}
   \includegraphics[scale=0.2]{immagini/hard_drive.jpg}
 \end{center}
\end{frame}

\begin{frame}
 \frametitle{Integrità dei dati}
 \begin{center}
   \huge No penne USB!
 \end{center}
 \begin{center}
   \includegraphics[scale=0.2]{immagini/penna_usb.jpg}
 \end{center}
\end{frame}


\begin{frame}
 \frametitle{Distruzione dati}
   \vspace{-0.2cm}
    \begin{center}
     \includegraphics[scale=0.6]{immagini/distruzione_dati.jpg}
    \end{center}
\end{frame}

\begin{frame}
 \frametitle{Distruzione dati}
 \begin{itemize}
  \item \textbf{Linux}: shred
  \item \textbf{Mac OS}: srm
  \item \textbf{Windows}: Free File Shredder
 \end{itemize}
\end{frame}

\begin{frame}
 \frametitle{il mio computer è lento}
 \begin{center}
   \includegraphics[scale=0.15]{immagini/slow_computer.jpg}
  \end{center}
\end{frame}

\begin{frame}
 \frametitle{Virus}
  \begin{center}
   \includegraphics[scale=0.15]{immagini/cavallo_troia.jpg}
  \end{center}
 
 I malware, virus non servono a distruggere il computer, bensì a sfruttarne le  risorse per scopi illeciti
\end{frame}


\begin{frame}
 \frametitle{Processore computer (CPU)}
 \begin{center}
   \includegraphics[scale=0.27]{immagini/cpu-ryzen.jpg}
  \end{center}
\end{frame}

{
\usebackgroundtemplate{\centering\includegraphics[height=\paperheight]{immagini/mips.jpg}}
\setbeamertemplate{navigation symbols}{}
\begin{frame}[plain]
\end{frame}
}

\begin{frame}
 \frametitle{Attacchi buffer overflow}
   \begin{center}
   \includegraphics[scale=0.14]{immagini/stack_overflow.jpg}
  \end{center}
\end{frame}

\begin{frame}[fragile]
 \frametitle{Attacchi buffer overflow}
\begin{verbatim}

 mele disponibili = 0
 se (mele disponibili == 0)
 {
    compra delle mele
    esci dal programma
 }
 altrimenti
 {
    mangia una mela
    esci dal programma
 }
\end{verbatim}
\end{frame}

\begin{frame}
 \frametitle{Attacchi buffer overflow}
 COSA IL COMPUTER HA IN MEMORIA
 \begin{center}
%\begin{adjustbox}{max width=\textwidth}
 \begin{tabular}{|c|c|c|}
\hline 
{NUMERO RIGA} & {COMANDO} & {PROGRAM COUNTER}\tabularnewline
{}&{}&{(riga seguente)}\tabularnewline
\hline 
\hline 
{1} & {mele disponibili = 0} & {}\tabularnewline
\hline
{2} & {se (mele disponibili == 0)} & {}\tabularnewline
\hline
{3} & {compra delle mele} & {}\tabularnewline
\hline
{4} & {esci dal programma} & {}\tabularnewline
\hline
{5} & {altrimenti} & {}\tabularnewline
\hline
{6} & {mangia una mela} & {}\tabularnewline
\hline
{7} & {esci dal programma} & {}\tabularnewline
\hline 
\end{tabular}
%\end{adjustbox}
\end{center}
\end{frame}

\begin{frame}
 \frametitle{Attacchi buffer overflow}
 COSA IL COMPUTER HA IN MEMORIA
 \begin{center}
%\begin{adjustbox}{max width=\textwidth}
 \begin{tabular}{|c|c|c|}
\hline 
{NUMERO RIGA} & {COMANDO} & {PROGRAM COUNTER}\tabularnewline
{}&{}&{(riga seguente)}\tabularnewline
\hline 
\hline 
{1} & {mele disponibili = 0} & {2}\tabularnewline
\hline
{2} & {se (mele disponibili == 0)} & {}\tabularnewline
\hline
{3} & {compra delle mele} & {}\tabularnewline
\hline
{4} & {esci dal programma} & {}\tabularnewline
\hline
{5} & {altrimenti} & {}\tabularnewline
\hline
{6} & {mangia una mela} & {}\tabularnewline
\hline
{7} & {esci dal programma} & {}\tabularnewline
\hline 
\end{tabular}
%\end{adjustbox}
\end{center}
\end{frame}

\begin{frame}
 \frametitle{Attacchi buffer overflow}
 COSA IL COMPUTER HA IN MEMORIA
 \begin{center}
%\begin{adjustbox}{max width=\textwidth}
 \begin{tabular}{|c|c|c|}
\hline 
{NUMERO RIGA} & {COMANDO} & {PROGRAM COUNTER}\tabularnewline
{}&{}&{(riga seguente)}\tabularnewline
\hline 
\hline 
{1} & {mele disponibili = 0} & {2}\tabularnewline
\hline
{2} & {se (mele disponibili == 0)} & {3}\tabularnewline
\hline
{3} & {compra delle mele} & {}\tabularnewline
\hline
{4} & {esci dal programma} & {}\tabularnewline
\hline
{5} & {altrimenti} & {}\tabularnewline
\hline
{6} & {mangia una mela} & {}\tabularnewline
\hline
{7} & {esci dal programma} & {}\tabularnewline
\hline 
\end{tabular}
%\end{adjustbox}
\end{center}
\end{frame}



\begin{frame}
 \frametitle{Attacchi buffer overflow}
 COSA IL COMPUTER HA IN MEMORIA
 \begin{center}
%\begin{adjustbox}{max width=\textwidth}
 \begin{tabular}{|c|c|c|}
\hline 
{NUMERO RIGA} & {COMANDO} & {PROGRAM COUNTER}\tabularnewline
{}&{}&{(riga seguente)}\tabularnewline
\hline 
\hline 
{1} & {mele disponibili = 0} & {2}\tabularnewline
\hline
{2} & {se (mele disponibili == 0)} & {3}\tabularnewline
\hline
{3} & {compra delle mele} & {4}\tabularnewline
\hline
{4} & {esci dal programma} & {}\tabularnewline
\hline
{5} & {altrimenti} & {}\tabularnewline
\hline
{6} & {mangia una mela} & {}\tabularnewline
\hline
{7} & {esci dal programma} & {}\tabularnewline
\hline 
\end{tabular}
%\end{adjustbox}
\end{center}
\end{frame}







\begin{frame}
 \frametitle{Attacchi buffer overflow}
 COSA IL COMPUTER HA IN MEMORIA
 \begin{center}
%\begin{adjustbox}{max width=\textwidth}
 \begin{tabular}{|c|c|c|}
\hline 
{NUMERO RIGA} & {COMANDO} & {PROGRAM COUNTER}\tabularnewline
{}&{}&{(riga seguente)}\tabularnewline
\hline 
\hline 
{1} & {mele disponibili = 0} & {2}\tabularnewline
\hline
{2} & {se (mele disponibili == 0)} & {3}\tabularnewline
\hline
{3} & {compra delle mele} & {4}\tabularnewline
\hline
{4} & {esci dal programma} & {FINE}\tabularnewline
\hline
{5} & {altrimenti} & {}\tabularnewline
\hline
{6} & {mangia una mela} & {}\tabularnewline
\hline
{7} & {esci dal programma} & {}\tabularnewline
\hline 
\end{tabular}
%\end{adjustbox}
\end{center}
\end{frame}















\begin{frame}[fragile]
 \frametitle{Attacchi buffer overflow}
\begin{verbatim}

 mele disponibili = 1  // CAMBIATO
 se (mele disponibili == 0)
 {
    compra delle mele
    esci dal programma
 }
 altrimenti
 {
    mangia una mela
    esci dal programma
 }
\end{verbatim}
\end{frame}

\begin{frame}
 \frametitle{Attacchi buffer overflow}
 COSA IL COMPUTER HA IN MEMORIA
 \begin{center}
%\begin{adjustbox}{max width=\textwidth}
 \begin{tabular}{|c|c|c|}
\hline 
{NUMERO RIGA} & {COMANDO} & {PROGRAM COUNTER}\tabularnewline
{}&{}&{(riga seguente)}\tabularnewline
\hline 
\hline 
{1} & {mele disponibili = 1} & {}\tabularnewline
\hline
{2} & {se (mele disponibili == 0)} & {}\tabularnewline
\hline
{3} & {compra delle mele} & {}\tabularnewline
\hline
{4} & {esci dal programma} & {}\tabularnewline
\hline
{5} & {altrimenti} & {}\tabularnewline
\hline
{6} & {mangia una mela} & {}\tabularnewline
\hline
{7} & {esci dal programma} & {}\tabularnewline
\hline 
\end{tabular}
%\end{adjustbox}
\end{center}
\end{frame}

\begin{frame}
 \frametitle{Attacchi buffer overflow}
 COSA IL COMPUTER HA IN MEMORIA
 \begin{center}
%\begin{adjustbox}{max width=\textwidth}
 \begin{tabular}{|c|c|c|}
\hline 
{NUMERO RIGA} & {COMANDO} & {PROGRAM COUNTER}\tabularnewline
{}&{}&{(riga seguente)}\tabularnewline
\hline 
\hline 
{1} & {mele disponibili = 1} & {2}\tabularnewline
\hline
{2} & {se (mele disponibili == 0)} & {}\tabularnewline
\hline
{3} & {compra delle mele} & {}\tabularnewline
\hline
{4} & {esci dal programma} & {}\tabularnewline
\hline
{5} & {altrimenti} & {}\tabularnewline
\hline
{6} & {mangia una mela} & {}\tabularnewline
\hline
{7} & {esci dal programma} & {}\tabularnewline
\hline 
\end{tabular}
%\end{adjustbox}
\end{center}
\end{frame}

\begin{frame}
 \frametitle{Attacchi buffer overflow}
 COSA IL COMPUTER HA IN MEMORIA
 \begin{center}
%\begin{adjustbox}{max width=\textwidth}
 \begin{tabular}{|c|c|c|}
\hline 
{NUMERO RIGA} & {COMANDO} & {PROGRAM COUNTER}\tabularnewline
{}&{}&{(riga seguente)}\tabularnewline
\hline 
\hline 
{1} & {mele disponibili = 1} & {2}\tabularnewline
\hline
{2} & {se (mele disponibili == 0)} & {5}\tabularnewline
\hline
{3} & {compra delle mele} & {}\tabularnewline
\hline
{4} & {esci dal programma} & {}\tabularnewline
\hline
{5} & {altrimenti} & {}\tabularnewline
\hline
{6} & {mangia una mela} & {}\tabularnewline
\hline
{7} & {esci dal programma} & {}\tabularnewline
\hline 
\end{tabular}
%\end{adjustbox}
\end{center}
\end{frame}

\begin{frame}
 \frametitle{Attacchi buffer overflow}
 COSA IL COMPUTER HA IN MEMORIA
 \begin{center}
%\begin{adjustbox}{max width=\textwidth}
 \begin{tabular}{|c|c|c|}
\hline 
{NUMERO RIGA} & {COMANDO} & {PROGRAM COUNTER}\tabularnewline
{}&{}&{(riga seguente)}\tabularnewline
\hline 
\hline 
{1} & {mele disponibili = 1} & {2}\tabularnewline
\hline
{2} & {se (mele disponibili == 0)} & {5}\tabularnewline
\hline
{3} & {compra delle mele} & {}\tabularnewline
\hline
{4} & {esci dal programma} & {}\tabularnewline
\hline
{5} & {altrimenti} & {6}\tabularnewline
\hline
{6} & {mangia una mela} & {}\tabularnewline
\hline
{7} & {esci dal programma} & {}\tabularnewline
\hline 
\end{tabular}
%\end{adjustbox}
\end{center}
\end{frame}

\begin{frame}
 \frametitle{Attacchi buffer overflow}
 COSA IL COMPUTER HA IN MEMORIA
 \begin{center}
%\begin{adjustbox}{max width=\textwidth}
 \begin{tabular}{|c|c|c|}
\hline 
{NUMERO RIGA} & {COMANDO} & {PROGRAM COUNTER}\tabularnewline
{}&{}&{(riga seguente)}\tabularnewline
\hline 
\hline 
{1} & {mele disponibili = 1} & {2}\tabularnewline
\hline
{2} & {se (mele disponibili == 0)} & {5}\tabularnewline
\hline
{3} & {compra delle mele} & {}\tabularnewline
\hline
{4} & {esci dal programma} & {}\tabularnewline
\hline
{5} & {altrimenti} & {6}\tabularnewline
\hline
{6} & {mangia una mela} & {7}\tabularnewline
\hline
{7} & {esci dal programma} & {}\tabularnewline
\hline 
\end{tabular}
%\end{adjustbox}
\end{center}
\end{frame}

\begin{frame}
 \frametitle{Attacchi buffer overflow}
 COSA IL COMPUTER HA IN MEMORIA
 \begin{center}
%\begin{adjustbox}{max width=\textwidth}
 \begin{tabular}{|c|c|c|}
\hline 
{NUMERO RIGA} & {COMANDO} & {PROGRAM COUNTER}\tabularnewline
{}&{}&{(riga seguente)}\tabularnewline
\hline 
\hline 
{1} & {mele disponibili = 1} & {2}\tabularnewline
\hline
{2} & {se (mele disponibili == 0)} & {5}\tabularnewline
\hline
{3} & {compra delle mele} & {}\tabularnewline
\hline
{4} & {esci dal programma} & {}\tabularnewline
\hline
{5} & {altrimenti} & {6}\tabularnewline
\hline
{6} & {mangia una mela} & {7}\tabularnewline
\hline
{7} & {esci dal programma} & {FINE}\tabularnewline
\hline 
\end{tabular}
%\end{adjustbox}
\end{center}
\end{frame}










\begin{frame}[fragile]
 \frametitle{Attacchi buffer overflow}
\begin{verbatim}

 mele disponibili = 2342343453453453454568  
 se (mele disponibili == 0)
 {
    compra delle mele
    esci dal programma
 }
 altrimenti
 {
    mangia una mela
    esci dal programma
 }
 DISTRUGGI SISTEMA
\end{verbatim}
\end{frame}

\begin{frame}
 \frametitle{Attacchi buffer overflow}
 COSA IL COMPUTER HA IN MEMORIA
 \begin{center}
%\begin{adjustbox}{max width=\textwidth}
 \begin{tabular}{|c|c|c|}
\hline 
{NUMERO RIGA} & {COMANDO} & {PROGRAM COUNTER}\tabularnewline
{}&{}&{(riga seguente)}\tabularnewline
\hline 
\hline 
{1} & {mele disponibili = 234234345345345345456} & {8}\tabularnewline
\hline
{2} & {se (mele disponibili == 0)} & {}\tabularnewline
\hline
{3} & {compra delle mele} & {}\tabularnewline
\hline
{4} & {esci dal programma} & {}\tabularnewline
\hline
{5} & {altrimenti} & {}\tabularnewline
\hline
{6} & {mangia una mela} & {}\tabularnewline
\hline
{7} & {esci dal programma} & {}\tabularnewline
\hline
{8} & {DISTRUGGI SISTEMA} & {}\tabularnewline
\hline
\end{tabular}
%\end{adjustbox}
\end{center}
\end{frame}


\begin{frame}
 \frametitle{Attacchi buffer overflow}
 COSA IL COMPUTER HA IN MEMORIA
 \begin{center}
%\begin{adjustbox}{max width=\textwidth}
 \begin{tabular}{|c|c|c|}
\hline 
{NUMERO RIGA} & {COMANDO} & {PROGRAM COUNTER}\tabularnewline
{}&{}&{(riga seguente)}\tabularnewline
\hline 
\hline 
{1} & {mele disponibili = 234234345345345345456} & {8}\tabularnewline
\hline
{2} & {se (mele disponibili == 0)} & {}\tabularnewline
\hline
{3} & {compra delle mele} & {}\tabularnewline
\hline
{4} & {esci dal programma} & {}\tabularnewline
\hline
{5} & {altrimenti} & {}\tabularnewline
\hline
{6} & {mangia una mela} & {}\tabularnewline
\hline
{7} & {esci dal programma} & {}\tabularnewline
\hline
{8} & {DISTRUGGI SISTEMA} & {FINE}\tabularnewline
\hline
\end{tabular}
%\end{adjustbox}
\end{center}
\end{frame}


\begin{frame}
 \frametitle{Wi-Fi e smart home}
 \begin{center}
    \includegraphics[scale=0.18]{immagini/homelynx.jpg}
 \end{center}
\end{frame}


{
\usebackgroundtemplate{\includegraphics[height=\paperheight,width=\paperwidth]{immagini/kismet.png}}
\setbeamertemplate{navigation symbols}{}
\begin{frame}[plain]
\end{frame}
}



\begin{frame}
 \frametitle{Smart tv}
  \begin{center}
   \includegraphics[scale=0.3]{immagini/smart_tv.jpg}
  \end{center}
\end{frame}




\section{Sicurezza nell'uso di internet}
\begin{frame}
 \begin{center}
  \huge Sicurezza nell'uso di internet
 \end{center}
\end{frame}

\begin{frame}
 \frametitle{Crittografia asimmetrica}
 Nasce dall'esigenza di trasportare dati protetti e la chiave di decodifica
 su un canale di comunicazione non sicuro.
   \begin{center}
   \includegraphics[scale=0.16]{immagini/crittografia.jpg}
  \end{center}
\end{frame}

\begin{frame}
 \frametitle{Crittografia asimmetrica}
 \huge Ogni persona possiede:
 %\begin{center}
 \begin{itemize}
  \item una chiave \textbf{privata};
  \item una chiave \textbf{pubblica}.
 \end{itemize} 
 %\end{center}
\end{frame}

\begin{frame}
 \frametitle{Crittografia asimmetrica}
\begin{adjustbox}{max width=\textwidth}
 \begin{tabular}{|c|c|c|c|c|c|}
\hline 
{TIPO} & {CODIFICA} & {DECODIFICA} & {FIRMA} & {VERIFICA} & {NOTE}\tabularnewline
{CHIAVE}&{MESSAGGI}&{MESSAGGI}&{MESSAGGI}&{FIRMA MITTENTE}&{}\tabularnewline
\hline 
\hline 
{Privata} & {} & {X} & {X} & {} & {Genera la} chiave pubblica\tabularnewline
{} & {} & {} & {} & {} & {chiave pubblica}\tabularnewline
\hline 
{Pubblica} & {X} & {} & {} & {X} & {Viene firmata da}\tabularnewline
{} & {} & {} & {} & {} & {altri utenti o da un'autorità} \tabularnewline
\hline 
\end{tabular}
\end{adjustbox}
\end{frame}

\begin{frame}
 \frametitle{Gestione password}
 \centering \huge KeepassXC
  \begin{center}
   \includegraphics[scale=2]{immagini/keepassxc.pdf}
  \end{center}
\end{frame}

\begin{frame}
 \frametitle{Generazione password}
 \begin{block}{Creare una password robusta}
  Si può partire da lettere di ogni parola di una frase
 \end{block}
 \pause
 nel mezzo del cammin di nostra vita mi ritrovai in una selva oscura, che la retta via
 era smarrita
\end{frame}

\begin{frame}
 \frametitle{Generazione password}
 \begin{block}{Creare una password robusta}
  Si può partire da lettere di ogni parola di una frase
 \end{block}
 \textbf{n}el \textbf{m}ezzo \textbf{d}el \textbf{c}ammin \textbf{d}i \textbf{n}ostra
 \textbf{v}ita \textbf{m}i \textbf{r}itrovai \textbf{i}n \textbf{u}na \textbf{s}elva
 \textbf{o}scura, \textbf{c}he \textbf{l}a \textbf{r}etta \textbf{v}ia \textbf{e}ra \textbf{s}marrita
 \begin{center}
  \huge Quindi: nmdcdnvmriusoclrves 
 \end{center}
\end{frame}

\begin{frame}
 \frametitle{Crack password}
 \begin{block}{Possibili combinazioni}
   \begin{equation*}
   10x10x10x10x10=10^{5}
   \end{equation*}
 \end{block}
  \begin{block}{Tempo per violazione}
  Assumendo un computer che faccia 1000 tentativi al secondo
   \begin{equation*}
   \frac{100.000\;combinazioni\;possibili}{1000\;tentativi/secondo}=100\;secondi
   \end{equation*}
 \end{block}
\end{frame}

\begin{frame}
 \frametitle{Furti di identità}
 \vspace{-0.5cm}
 \begin{center}
  \includegraphics[scale=0.2]{immagini/amici.jpg}
 \end{center}

\end{frame}


\begin{frame}
 \frametitle{Virus Zeus + furti carte di credito}
   \begin{center}
   \includegraphics[scale=0.2]{immagini/soldi.jpg}
 \end{center}
\end{frame}

\begin{frame}
 \frametitle{Virus Zeus + furti carte di credito}
   \begin{center}
    \includegraphics[scale=2]{immagini/mobile-banking.jpg}
   \end{center}
\end{frame}


\begin{frame}
 \frametitle{Protezione minori su internet}
  \begin{center}
   \includegraphics[scale=0.3]{immagini/Kaspersky-Safe-Kids.jpg}
 \end{center}
\end{frame}

\begin{frame}
 \frametitle{Protezione minori su internet}
  \begin{center}
   \includegraphics[scale=0.3]{immagini/Kaspersky-safe-kids-localizzazione.jpeg}
 \end{center}
\end{frame}

\begin{frame}
 \frametitle{Protezione minori su internet}
  \begin{center}
   \includegraphics[scale=0.3]{immagini/Kaspersky-safe-kids-amministrazione.jpeg}
 \end{center}
\end{frame}

\begin{frame}
 \frametitle{Protezione minori su internet}
  \begin{center}
   \includegraphics[scale=0.3]{immagini/Kaspersky-safe-kids-amministrazione-2.jpeg}
 \end{center}
\end{frame}

\begin{frame}
 \frametitle{Nei servizi online gratuiti, sei tu il prodotto}
 \vspace{-0.4cm}
   \begin{center}
   \includegraphics[scale=0.09]{immagini/supermarket.jpg}
 \end{center}
\end{frame}


\begin{frame}
 \frametitle{Geolocalizzazione globale Google}
 \begin{center}
  \includegraphics[scale=0.2]{immagini/map-celle.jpg}
 \end{center}
\end{frame}

{
\usebackgroundtemplate{\includegraphics[height=\paperheight,width=\paperwidth]{immagini/map-gps-celle.jpg}}
\setbeamertemplate{navigation symbols}{}
\begin{frame}[plain]
\end{frame}
}

{
\usebackgroundtemplate{\includegraphics[height=\paperheight,width=\paperwidth]{immagini/map-celle.jpg}}
\setbeamertemplate{navigation symbols}{}
\begin{frame}[plain]
\end{frame}
}

{
\usebackgroundtemplate{\includegraphics[height=\paperheight,width=\paperwidth]{immagini/map-celle-wifi_grande.jpg}}
\setbeamertemplate{navigation symbols}{}
\begin{frame}[plain]
\end{frame}
}

{
\usebackgroundtemplate{\includegraphics[height=\paperheight,width=\paperwidth]{immagini/map-interni.jpg}}
\setbeamertemplate{navigation symbols}{}
\begin{frame}[plain]
\end{frame}
}

{
\usebackgroundtemplate{\includegraphics[height=\paperheight,width=\paperwidth]{immagini/nsa-1.jpg}}
\setbeamertemplate{navigation symbols}{}
\begin{frame}[plain]
\end{frame}
}

{
\usebackgroundtemplate{\includegraphics[height=\paperheight,width=\paperwidth]{immagini/nsa-2.jpg}}
\setbeamertemplate{navigation symbols}{}
\begin{frame}[plain]
\end{frame}
}

{
\usebackgroundtemplate{\includegraphics[height=\paperheight,width=\paperwidth]{immagini/nsa-3.jpg}}
\setbeamertemplate{navigation symbols}{}
\begin{frame}[plain]
\end{frame}
}

{
\usebackgroundtemplate{\includegraphics[height=\paperheight,width=\paperwidth]{immagini/nsa-4.jpg}}
\setbeamertemplate{navigation symbols}{}
\begin{frame}[plain]
\end{frame}
}

{
\usebackgroundtemplate{\includegraphics[height=\paperheight,width=\paperwidth]{immagini/nsa-5.jpg}}
\setbeamertemplate{navigation symbols}{}
\begin{frame}[plain]
\end{frame}
}

{
\usebackgroundtemplate{\includegraphics[height=\paperheight,width=\paperwidth]{immagini/nsa-utah.jpg}}
\setbeamertemplate{navigation symbols}{}
\begin{frame}[plain]
\end{frame}
}

{
\usebackgroundtemplate{\includegraphics[height=\paperheight,width=\paperwidth]{immagini/nsa-utah-vista-aerea.jpg}}
\setbeamertemplate{navigation symbols}{}
\begin{frame}[plain]
\end{frame}
}

\begin{frame}
 \frametitle{Internet... è per sempre}
  \begin{center}
   \includegraphics[scale=0.3]{immagini/nsa-utah-vista-aerea.jpg}
 \end{center}
\end{frame}

\begin{frame}
 \frametitle{Internet... è per sempre}
 \begin{block}{Centro di stoccaggio dati National Security Agency (Utah, USA)}
  Lo Utah Data Center, noto anche come Intelligence Community Comprehensive
  National Cybersecurity Initiative Data Center, è un impianto di stoccaggio
  di dati per la United States Intelligence Community, progettato per
  memorizzare dati nell'ordine degli exabyte.
 \end{block}

  \begin{center}
   \includegraphics[scale=0.2]{immagini/nsa-utah-vista-aerea.jpg}
 \end{center}
\end{frame}

\end{document}

% SICUREZZA DISPOSITIVI ELETTRONICI
% 

% SICUREZZA NELL'UTILIZZO DI INTERNET
% delinquenti privi di capacità informatiche potrebbero pagare qualcuno per svolgere attività illecite
% facebook like buttons